\chapter{Conclusion} \label{chap:conclusion}

We conclude with a brief summary of the work conducted, the developed implementation and results found.
Additionally, we provide a discussion on avenues for future work.

This work has looked at recent developments in the spaces of transport layer protocols and programming languages.
We have aimed to analyse if the technologies, QUIC and Rust, set to succeed current industry standards can be used for IoT network firmware.
The context of IoT provides interesting constraints in terms of the balance of performance, security and hardware constraints.
We have discussed MQTT - the most widely used IoT application layer protocol.
MQTT presents an HTTP like lightweight, publish-subscribe communication system.
We have also shown that MQTT relies on the transport layer to provide transport semantics and secure communication.

By analysing the currently available QUIC and MQTT implementations, we identified libraries that were used as the basis for our implementation.
Both of the selected libraries were developed in the Rust programming language.
Using $Quinn$, we developed an intermediate API - $QuicSocket$ that provides an API for connecting, sending and receiving using QUIC.
Further, we have changed the base $rumqtt$ implementation to use QUIC using this API, resulting in $MQuicTT$ - a QUIC based MQTT implementation.

Using a mininet test bench, we have analysed the performance of the the resulting implementation and have shown that it is comparable to $rumqtt$ in terms of connection time and total transmission time.

We have analysed the binary sizes of the broker and client of $MQuicTT$ and identified the dependencies that contribute the most to the size.
We have shown that Rust's use of $utf-8$ strings can result in size bloat.
Additionally, we have further analysed the contributions made by the QUIC and TLS implementations to devise a methodology for trimming down the QUIC stack.

Overall, we have found that while $MQuicTT$ performed comparably to $rumqtt$, we could not trim down the binary size enough for it to be used on truly hardware constrained devices.
However, the resulting implementation can be used on devices of the Raspberry Pi class.

\section{Future work} \label{sec:future_work}

This section will address the limitations of the work that we have done and discuss future work that would improve or expand upon the topic.
Each section will present a direction for further work and a discussion of it.

\subsection{QUIC implementations}

The number of available QUIC implementations has already far surpassed the number of major TCP implementations that are used in deployments.
Even during the course of this work being produced, a new Rust QUIC implementation was made public - Amazon's $s2n-quic$~\footnote{\url{https://github.com/aws/s2n-quic}}.
The number of QUIC implementations and the difference in direction compared to TCP has several implications on this work.

Firstly, a new more efficient implementation of Rust could be developed that would be a better fit as the base implementation for $MQuicTT$.
Hence, new implementations would need to be analysed.

Secondly, each new implementation means that many servers will have to interface with differing implementations, often with different implementation semantics.
An analysis of how different QUIC implementations perform in conjunction with each other could provide different efficiency statistics to the ones presented in our analysis.
That is, both our broker and clients used $Quinn$ and it would be interesting to analyse if using, for example, $Quinn$ for the broker and $s2n-quic$ for the client, impacts performance.

Additionally, with so many implementations having to interface with each other, it would be worthwhile to analyse the number of bugs in QUIC deployments compared to TCP ones.

\subsection{Binary analysis}

As discussed, all binary analysis for this work was conducted manually.
This task is rather tedious and error prone.
While it is possible to create a dependency analyser and a decompiler to analyse the binaries, it is currently impossible to connect this automatically with different features of a protocol.

For this to be possible we would need to augment protocol implementations with some sort of indication of a feature.
In the Rust programming language this could be done via a macro.
For example, a function with the macro $\#[feature(VersionNegotation)]$ could be used to indicate Quic's version negotiation feature.
This method, however, can be viewed as quite restrictive as it essentially enforces a by-feature abstraction.
As we have seen in $Quinn$, this is not always the way developers want to write their code.

Hence, to be able to conduct this sort of analysis we would need to find an nonintrusive way of augmenting implementations with such details.

\subsection{Comparison analysis to CoAP}

We have previously discussed the similarities between using QUIC for MQTT and using CoAP.
Both methods utilise UDP and an HTTP style messaging approach.
Additionally, both provide retalitevly low code overhead, which is important for hardware constrained devices.

While this work did not focus on the comparison of $MQuicTT$ to CoAP, such an analysis would be crucial to understanding the performance of QUIC in IoT.
An additional avenue of future work is extracting the core features that both these approaches share and analysing if they could be employed to make other protocols better for hardware constrained devices.
The result of such work could create a general approach to creating protocols for IoT devices.

\subsection{Improved test bench}

There are several avenues that could be taken for improving the analysis test bench that we used.
Firstly, a remote controller could be implemented for the mininet topology to be able to use an actual mesh topology, with cycles, instead of the minimum spanning tree one deployed for this work.
This would also mean that different load balancing and routing options could be tested for the scenarios, to better replicate a realistic deployment.
This would also mean that network failures, such as switches going offline, could be simulated programmatically.
Naturally, a hardware test bench analysis would also improve any further work.