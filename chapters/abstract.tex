\begin{abstract}
    New advances in networked systems and programming languages are set to succeed existing industry standards.
    In the transport layer space, QUIC - a is set to succeed the TCP/TLS stack and is advantageous for performance and security by leveraging improved features such as a simpler, secure handshake, and streams.
    While in the programming languages space, Rust, a new systems programming language, uses a strong type system to guarantee memory-safety and deadlock-freedom, while still being performant.
    In this work we analyse the feasibility of using a Rust based QUIC implementation as the basis for secure communication in IoT devices.
    To do so, in this work we develop $MQuicTT$ - a Rust based MQTT implementation using QUIC at the transport layer.
    We compare the performance of the resulting solution with existing MQTT implementations in various use cases and analyse the hardware resources needed to deploy it.
    We find that the performance of the implementation is on par with the baseline.
    Finally, we discuss the binary size of the implementation and the possible methods to generally reduce Rust binary sizes for IoT network firmware using QUIC.
\end{abstract}