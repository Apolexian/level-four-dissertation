\section{QuicSocket}\label{chapter:quic_socket}

This section will present $QuicSocket$~\footnote{QuicSocket - \url{https://github.com/Apolexian/QuicSocket}} - the intermediate wrapper API that we developed using $Quinn$ to facilitate establishing a QUIC connection, control streams, and send and receive data.

$QuicSocket$ provides two main types to communicate with $Quinn$ - a $QuicServer$ and a $QuicClient$, and a trait that both of these implement - $QuicSocket$.
The $QuicSocket$ trait defines three function: $new$, $send$ and $recieve$. 
When initialised using the $new$ function, the server and the client attempt to open a QUIC connection.
In the case of the server, it binds to a given port and starts listening for a connection request.
The client binds to a random available port and attempts to connect to a provided URL.
Additionally, in the $new$ function, the server generates a new TLS certificate or reads it from a specified path.
The client's $new$ function must be supplied with the corresponding certificate's path.
If either certificate is not present, the server will reject the connection attempt.
Once a connection has been opened, data transfer can happen.

\begin{lstlisting}[language=Rust, caption={The $send$ function that the $QuicClient$ and $QuicServer$ use for sending data. Data transfer is facilitated by opening a bidirectional stream on the pre-existing connection.}, label=lst:quicsocket_send]
    async fn send(&mut self, payload: Vec<u8>) -> Result<()> {
        let (mut send, _) = self
            .connection
            .open_bi()
            .await
            .map_err(|e| anyhow!("failed to open stream: {}", e))
            .unwrap();
        send.write_all(&payload)
            .await
            .map_err(|e| anyhow!("failed to send request: {}", e))?;
        send.finish()
            .await
            .map_err(|e| anyhow!("failed to shutdown stream: {}", e))?;
        Ok(())
    }
\end{lstlisting}

In order to send data, either side can use the $send$ function demonstrated in Listing~\ref{lst:quicsocket_send}.
We first open a bidirectional stream and then write the provided payload.
Once the client completes writing the payload, it shuts down the stream.
On the other side, the receive function in Listing~\ref{lst:quicsocket_recv} processes the next available QUIC stream and reads the data on it into a supplied buffer.
It then returns the number of bytes written to the buffer.
Hence, to use $QuicSocket$, we must either provide or generate TLS certificates, open a QUIC connection and then use the send and receive functions.
The developed API is analogous to how a TCP socket sends and receives data.
An alternative approach that we considered was to make the $recv$ function return the value that the client or server read from the stream.
However, this approach does not integrate well into any protocol implementation due to the general preference for the buffer pattern.

\begin{lstlisting}[language=Rust, float, caption={A similar application in Rust will not compile due to the safety guaranteed by the ownership system. A borrow occurs when $example\_ref$ is assigned to point to $new\_example$, however the ownership system recognises that the borrowed value does not live long enough.}, label=lst:quicsocket_recv]
    async fn recv(&mut self, buf: &mut [u8]) -> Result<usize> {
        let (_, mut recv) = self.bi_streams.next().await.unwrap().unwrap();
        let len = recv
            .read(buf)
            .await
            .map_err(|e| anyhow!("failed to read response: {}", e))?;
        Ok(len.unwrap())
    }
\end{lstlisting}

Error handling is handled via the $anyhow$ package throughout the library for idiomatic error handling.
In addition to this, we wrap return values in Rust's $Result$ construct.
Hence, all errors are idiomatically mapped wherever a failure can occur.

An important design choice that we have made when developing $QuicSocket$ is using QUIC streams.
Every call to send and receive in the current implementation creates a bidirectional stream.
An alternate choice was to open a stream at the start of the connection and use it until the connection closes.
However, the latter approach runs into issues when it comes to reading from the stream.
All data was read on one call to receive, which unnaturally coalesced MQTT packets.

Notably, all operations happen asynchronously.
However, as of the implementation time, the Rust standard library does not support asynchronous traits.
The workaround to this issue is using the $async-trait$ crate. 
An unfortunate drawback of using this crate is that every function call results in a heap allocation due to the crate's implementation semantics.
The extra heap allocations do not usually present an issue; however, in the case of hardware constrained devices, this may lead to a bottleneck depending on the usage of the library.
The reasons for not supporting asynchronous traits in Rust are somewhat complex.
For one, an asynchronous function in Rust returns an $impl Future$, meaning that an asynchronous trait would have to support returning $impl Trait$, which it does not.
However, the $async-trait$ crate instead returns a $dyn Future$, the Rust implementation of dynamic dispatch, resulting in a heap allocation but allowing it to be used inside of traits.
We will not further discuss the other reasons for this crate's existence; however,~\cite{matsakis_baby_2019} created a comprehensive analysis of this issue.
Once Rust incorporates asynchronous traits into the language, the heap allocation issue will be resolved, resulting in this problem no longer existing for hardware constrained devices.
