\chapter{MQuicTT} \label{chap:port}

This section describes how we used $QuicSocket$ to create a QUIC port of $rumqtt$, the design decisions that we have taken, and possible alternate approaches.

As previously mentioned, $rumqtt$ provides to APIs: $rumqttc$ for creating MQTT clients and $rumqttd$ for creating brokers.
Hence, each of these APIs relies on an underlying TCP implementation to transmit data.
A configuration file is also required for $rumqtdd$ that specifies various parameters such as paths to TLS keys and the ports that MQTT will operate on.
Additionally, $rumqtt$ implements its own layer of TLS to ensure secure communication on top of the TLS layer provided by the underlying transport protocol.

To create $MQuicTT$, we have taken steps that can broadly be summarised in two phases: altering the default TLS code and changing the underlying transport protocol from TCP to QUIC.

As we are working in the domain of hardware constrained devices, providing multiple layers of encryption would be excessive, even for the goal of secure communication.
We can afford for only the transport layer to handle encryption.
Hence, the first step that we have taken is to remove all TLS functionality from both $rumqttd$ and $rumqttc$.

After this, we identified that both APIs have a central interface that controls network communication.
In the case of $rumqttc$ this is the $network$ struct inside $framed.rs$~\footnote{framed.rs - \url{https://github.com/bytebeamio/rumqtt/blob/master/rumqttc/src/framed.rs}} and in the case of $rumqttd$ it is the $network$ struct inside $network.rs$~\footnote{network.rs - \url{https://github.com/bytebeamio/rumqtt/blob/master/rumqttd/src/network.rs}}.
Both of these interfaces use a $tcpstream$ to send and receive data.
The $tcpstream$ is opened when the struct is initialised and closed when the MQTT connection ends.
As TCP is a stateful protocol, minimal connection handling occurs after opening the initial stream.

\begin{lstlisting}[language=Rust, caption={An example of initialising an MQuicTT client. The QUIC connection is established by initialising a $QuicClient$ and the resulting client is passed as an MQTT option.}, label=lst:MQuicTT:client]
    #[tokio::main(worker_threads = 1)]
    async fn main() {
        ...
        let quic_client = QuicClient::new(None).await;
        let mut mqttoptions = MqttOptions::new("test-1", ip, 1883, addr, quic_client);
        mqttoptions.set_connection_timeout(10);
        mqttoptions.set_keep_alive(5);
    
        let (client, mut eventloop) = AsyncClient::new(mqttoptions, 100);
        task::spawn(async move {
            requests(client).await;
        });
    }
\end{lstlisting}

As we have modelled $QuicSocket$ to have an API that closely resembles a typical $tcpstream$ API, the challenge of this stage of the port was managing QUIC streams and the state of connection.
Due to UDP, the protocol underlying QUIC, being stateless, we have handled the connection differently in brokers and clients.

The connection request is sent before an MQTT client object is initialised in the client's case.
Upon successful connection, the programmer must pass the connection to the initialisation function of the client.
This means that a single QUIC connection underpins the client's entire MQTT connection, and streams are used for packets.
Hence, when the client wishes to close the MQTT connection, it also closes the underlying QUIC connection.
In order to do this we have modified the parameters needed to create an $rumqtt$ MQTT client.
An example of the initialisation of a QUIC connection and MQTT client using our implementation is demonstrated in Listing~\ref{lst:MQuicTT:client}.

In the case of a broker, we must wait for a client to send a connection request.
Hence, we may initialise the broker and wait for a QUIC connection, only allowing data transfer when any connection is established.
This also means that the broker does not need to track which connection comes from which client.

\begin{lstlisting}[language=Rust, caption={An example of initialising an MQuicTT broker. We can see that no operations with $QuiSocket$ are required for this initialisation as all QUIC operations are handled internally.}, label=lst:MQuicTT:broker]
    fn main() {
        pretty_env_logger::init();
        let config: Config = confy::load_path("rumqttd.conf").unwrap();
        let mut broker = Broker::new(config);

        let mut tx = broker.link("localclient").unwrap();
        thread::spawn(move || {
            broker.start().unwrap();
        });
        ...
    }

\end{lstlisting}

Hence, as we can see in Listing~\ref{lst:MQuicTT:broker}, the programmer is not required to use any $QuicSocket$ code when creating an $MQuicTT$ broker.

We considered an alternate approach to refactoring the $rumqtt$ codebase to consolidate all network code into a shared internal package.
This would mean that the API for creating a client and broker would be identical and perhaps more ergonomic.
However, we decided not to go with this approach because $rumqttc$ and $rumqttd$ would no longer be disjoint, independent components.
With the chosen approach, it is possible to use different underlying transport protocols for either side, which allows for flexibility in implementation.

Another avenue of discussion stems from $rumqtt$ having two different client implementations: an asynchronous client and an asynchronous client.
In either case, a $tokio$ eventloop is used to handle events.
As $QuicSocket$ requires an asynchronous environment, the natural choice was to use the asynchronous client.
It could however be possible to handle asynchronous events more efficiently by tying the QUIC eventloop to the MQTT eventloop as in the approach described by~\cite{kumar_implementation_2019}.
In our case, we leave the comparison between these two approaches as an avenue for future work.
