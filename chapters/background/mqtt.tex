\section{MQTT}

Considering the constraints that apply to IoT devices, we will now look at one of the widely used application-level IoT protocols.
MQTT~\citep{oasis_mqtt_2014}, originally standing for Message Queuing Telemetry Transport, is designed to be a lightweight protocol to transport messages between devices using the publish-subscribe method.
MQTT defines two types of participants - the broker and the client.
The broker is a server that receives messages from all clients and then routes these messages to the appropriate destinations.
On the other hand, a client is a device that runs an MQTT library and sends the broker messages.

The broker handles the routing of messages in MQTT via \textit{topics}.
When a client wishes to publish a message, it does so on a client, and the broker distributes this message to all other clients subscribed to this topic.
The broker facilitating communication means that the publishing client does not need to keep track of other clients' locations to communicate with them.

Communication via MQTT can happen after the initial connection request from the client and the subsequent connection acknowledgement from the broker.
The connection request can specify a quality of service, referred to as the QoS parameter, to indicate the nature of message delivery.

\begin{figure}[ht]
    \begin{center}
        \begin{tikzpicture}[>=latex]
            \coordinate (A) at (2,12);
            \coordinate (B) at (2,0);
            \coordinate (C) at (8,12);
            \coordinate (D) at (8,0);
            \coordinate (J) at (14,12);
            \coordinate (K) at (14,0);
            \draw[thick] (A)--(B) (C)--(D) (J)--(K);
            \draw (A) node[above]{\Large Client A};
            \draw (C) node[above]{\Large Broker};
            \draw (J) node[above]{\Large Client B};

            \coordinate (E) at ($(A)!.1!(B)$);
            \draw (E) node[left]{\begin{tabular}{r}
                \end{tabular}};

            \coordinate (F) at ($(C)!.3!(D)$);
            \draw (F) node[right]{\begin{tabular}{l}
                \end{tabular}};
            \draw[->] (E) -- (F) node[midway,sloped,above]{\verb$Connect, QoS=0$};

            \coordinate (G) at ($(A)!.5!(B)$);
            \draw (G) node[left]{};
            \draw[->] (F) -- (G) node[midway,sloped,above]{\verb$Connack$};

            \coordinate (H) at ($(A)!.5!(B)$);
            \draw (H) node[right]{};

            \coordinate (I) at ($(C)!.7!(D)$);
            \draw (I) node[left]{};
            \draw[->] (H) -- (I) node[midway,sloped,above]{\verb$Subscribe Topic A$};

            \coordinate (H) at ($(C)!.8!(D)$);
            \draw (H) node[right]{};

            \coordinate (I) at ($(A)!.9!(B)$);
            \draw (I) node[left]{};
            \draw[->] (H) -- (I) node[midway,sloped,above]{\verb$Publish Message B$};

            \coordinate (L) at ($(J)!.2!(K)$);
            \draw (E) node[left]{\begin{tabular}{r}
                \end{tabular}};

            \coordinate (M) at ($(C)!.4!(D)$);
            \draw (L) node[right]{\begin{tabular}{l}
                \end{tabular}};
            \draw[->] (L) -- (M) node[midway,sloped,above]{\verb$Connect, QoS=0$};

            \coordinate (L) at ($(C)!.4!(D)$);
            \draw (E) node[left]{\begin{tabular}{r}
                \end{tabular}};

            \coordinate (M) at ($(J)!.6!(K)$);
            \draw (L) node[right]{\begin{tabular}{l}
                \end{tabular}};
            \draw[->] (L) -- (M) node[midway,sloped,above]{\verb$Connack$};

            \coordinate (L) at ($(J)!.6!(K)$);
            \draw (E) node[left]{\begin{tabular}{r}
                \end{tabular}};

            \coordinate (M) at ($(C)!.75!(D)$);
            \draw (L) node[right]{\begin{tabular}{l}
                \end{tabular}};
            \draw[->] (L) -- (M) node[midway,sloped,above]{\verb$Publish Message B, Topic A$};
        \end{tikzpicture}
        \caption{The MQTT connection, specifying the QoS measure, followed by the publishing of a message to a broker. It is important to note that the QoS measure does not impact the underlying data transmission provided by a protocol such as TCP. In this example, client B publishes message B to topic A after connecting, and the broker publishes this message to client A as it previously subscribed to topic A.}
        \label{fig:mqtt}
    \end{center}
\end{figure}

The possible QoS parameters are as follows:

\begin{itemize}
    \item At most once (fire and forget) - a client sends a message once, and the broker takes no steps to acknowledge delivery.
    \item At least once (acknowledged deliver) - a message is re-sent until the broker acknowledges it has received it.
    \item Exactly once (assured delivery) - the client and broker have to participate in a two-way handshake to ensure that a single copy of a message is received.
\end{itemize}

We present an example of an MQTT connection with two clients and a broker with QoS of at most once in Figure~\ref{fig:mqtt}.
With each level of QoS measure increasing, so does the communication overhead.
However, this measure only affects the MQTT part of the communication.
The underlying transport layer protocol, such as TCP, will still act as intended.
Hence, MQTT relies on an underlying transport-level protocol for data transmission.
Additionally, MQTT sends all its connection credentials in plain text format; hence, the communication is vulnerable if the transport layer does not provide encryption.
The most widely used suite for providing data transfer and encryption for MQTT is TCP/TLS; however, any protocol that provides lossless, bi-directional communication can be used.

As QUIC provides such a form of communication, we can see that we can use it as the transport level protocol for MQTT.
The benefits of this are numerous.
For example, perceived performance benefits from lower communication overhead and encryption at header and packet levels.
QUIC also comes with the benefit of multiplexing.
QUIC can open several streams on a single connection instead of opening several connections from one client.
Multiplexing will later play an essential role in our implementation of MQTT/QUIC.

The first implementation of MQTT using QUIC presented by~\cite{kumar_implementation_2019} uses the $ngtcp2$ implementation of QUIC in the C programming language.
The authors find that QUIC reduces the connection overhead time significantly and reduces the processor and memory usage.

Due to the authors' chosen underlying QUIC implementation, their API implementation requires its own message queue amongst other common functions such as key settlement.
The QUIC implementation chosen for this project allows the library to handle some of this logic instead.
We justify our choice of QUIC library and compare existing implementations in chapter~\ref{section:quic_impl}.