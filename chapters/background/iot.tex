\section{The Internet of Things}

There is no set definition for what constitutes an IoT device. 
However, an IoT device is generally a small chip that possesses some processing power and may have embedded sensors.
The key aspect of IoT devices is that they facilitate data exchange with other devices and systems over the Internet.
The modern version of IoT can be attributed to\citepos{weiser_computer_1991} work on ubiquitous computing, although the term itself first appeared in a speech by Peter T. Lewis in 1985.
IoT has applications in various fields, including smart home automation, healthcare, consumer applications, etc.

In terms of classifications within networking and IoT technologies, we can generally split them into wireless and wired, with the former split into short-range, medium-range, and long-range.

Short-range wireless IoT technologies include Bluetooth mesh networks, Z-wave, ZigBee and Wi-Fi, and other lesser-used technologies.
These technologies are the most prevalent in consumer IoT devices, such as smart home applications.
Medium range networks are used heavily in mobile devices with technologies such as LTE and 5G.
The technologies again present an interest due to the amount of traffic that the Internet sees from mobile devices.
On the other hand, long-range networks are rather specific in their applications; for example, VSAT - a satellite communication technology that uses small dish antennas is not something we would necessarily think of when encountering IoT.
Due to the limited application of long-range technologies, we opted to leave them out of our analysis.

Ethernet remains the dominant general-purpose networking standard in terms of wired technologies used by IoT devices.
Although wired technologies provide advantages in terms of data transfer speed, they limit deployments due to the physical wiring constraints.

Having defined what we mean by an IoT device in a network, we now consider the constraints that apply to these devices.
Due to the use cases of IoT, the form factor of these devices should be small.
For example, a processing unit inside a home assistant has to fit in its enclosure.
Additionally, many of these devices have to run for long periods, sometimes on a single lithium battery, hence needing to consume as least energy as possible.
Many use cases of IoT devices also require many of them connected in a mesh network.
For example,~\cite{ericsson_iot_2018} estimated that $0.5$ connected devices were used per square meter in a smart factory, with demand growing.
Large scale deployments add economic constraints to IoT devices as they need to be manufactured from relatively cheap components.

These constraints mean that IoT devices are limited in hardware resources.
Hardware limitations come in three primary forms - CPU power, memory and storage.
Storage in the form of flash memory provides the hardest to solve problems regarding secure data transfer.
The keys required for protocols such as TLS are often large and need to be stored.
For example, the $ESP8266$ controller, a widely used IoT chip, comes with $4$Mb of flash memory.
After installing the firmware and binaries needed, little to no memory may remain for additional storage.

The circumvention of these constraints at the cost of insecure firmware and communication, amongst other issues, is why IoT has become synonymous with security concerns.
Efforts to classify the security issues in the IoT space~\citep{alaba_internet_2017,gupta_security_2021,swamy_security_2017} and create a taxonomy have generally shown several main topics: insecure firmware level code, issues with privacy due to authentication and authorisation and general security concerns due to poor encryption at the transport layer.

Insecure firmware in IoT devices comes from issues with firmware updates and general vulnerabilities stemming from code.
A primary reason is that most programmers opt to create software for IoT devices in memory unsafe languages.
Languages such as C provide the needed efficiency to circumvent processing constraints; however, they also leave room for memory management issues, leading to vulnerabilities.
When it comes to privacy, the primary goal is ensuring data integrity and confidentiality.
Data sent via the network must not be tampered with nor snooped on by third parties during communication.
Man in the middle attacks is a prime concern for protocols such as MQTT due to their lack of self-imposed encryption.

Hence, finding a way to circumvent the hardware constraints presented by IoT devices and still provide secure data transfer is paramount to the safe adoption of IoT.
Additionally, opting to create IoT firmware code in a memory safe language may prevent vulnerabilities that are present on IoT devices from the moment of deployment.
