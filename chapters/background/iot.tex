\section{The Internet of Things}

It is hard to give a set criterion or definition for which devices qualify as IoT devices.
Generally, an IoT device usually possesses some processing power and may have embedded sensors.
The key aspect of IoT devices is that they facilitate data exchange with other devices and systems over the Internet.
The modern version of IoT can be attributed to\citepos{weiser_computer_1991} work on ubiquitous computing, although the term itself first appeared in a speech by Peter T. Lewis in 1985.
IoT has many applications in various fields, including smart home automation, healthcare, consumer applications, etc.

In terms of classifications within networking and IoT technologies, we can generally split them into wireless and wired, with the former split into short-range, medium-range, and long-range.

Short-range wireless IoT technologies include Bluetooth mesh networks, Z-wave, ZigBee and Wi-Fi, and other lesser-used technologies.
Due to the inherent advantages of short-range wireless communication in IoT applications such as smart homes, this category of IoT technologies was the primary focus for the project, as discussed in later chapters.
Medium range networks are used heavily in mobile devices with technologies such as LTE and 5G.
The technologies again present an interest due to the amount of traffic that the Internet sees from mobile devices.
On the other hand, long-range networks are rather specific in their applications, for example, VSAT - a satellite communication technology that uses small dish antennas.
Due to the limited application of long-range technologies, we opted to leave them out of our analysis.

Ethernet remains the dominant general-purpose networking standard in terms of wired technologies used by IoT devices.
Although wired technologies provide advantages in terms of data transfer speed, they limit deployments due to the physical wiring constraints.

Due to the uses of IoT, the form factor of these devices has to be physically small.
Many of these devices have to run for long periods on a single lithium battery, hence needing to consume as least energy as possible.
Additionally, many use cases of IoT devices require a large number of them connected in a network.
For example,~\cite{ericsson_iot_2018} estimated that $0.5$ connected devices were used per square meter in a smart factory, with demand growing.
Large scale deployments add economic constraints to IoT devices as they need to be manufactured from relatively cheap components.

These constraints mean that IoT devices are limited in hardware resources.
Hardware limitations come in three primary forms - CPU power, memory and storage.
Storage in the form of flash memory provides the hardest to solve problems regarding secure data transfer.
The keys required for protocols such as TLS are often large and need to be stored.
For example, the $ESP8266$ controller, a widely used IoT chip, comes with $4$Mb of flash memory.
After installing the firmware and binaries needed for the device to perform its function, little to no memory may remain for additional storage.

Efforts to classify the security issues in the IoT space~\citep{alaba_internet_2017,gupta_security_2021,swamy_security_2017} and create a taxonomy have generally shown several main topics: issues with privacy due to authentication and authorisation and general security concerns due to poor encryption at the transport layer.

Insecure firmware in IoT devices comes from issues with firmware updates and generally insecure code.
Most programmers opt to create software for IoT devices in the C programming language, which, while providing the needed efficiency, is also a source of insecure code that can lead to potential attacks, such as buffer overflows.

On the other hand, we must ensure data integrity and confidentiality to ensure privacy and general security.
Data sent via the network must not be tampered with nor snooped on by third parties during communication.
Hence, privacy requires secure methods for authentication, authorisation and transport-level encryption.

Hence, finding a way to circumvent the hardware constraints presented by IoT devices and still provide secure data transfer is paramount to the safe adoption of IoT.