\chapter{Analysis} \label{chapter:eval}

This section presents the analysis results conducted using the previously described methodology.
The analysis is broadly split into two stages: performance analysis and binary size analysis.
The experiment design for the performance analysis can be found in Section~\ref{chap:net_sim}, and the experiment design for the binary analysis can be found in Section~\ref{sec:exp_bin}.
The performance analysis is further split into a connection time analysis and a general transmission time analysis.

\chapter{Network simulation}

When evaluating the network performance of the implementations, we considered two options: using real IoT devices or using a network simulation tool.
Due to technical limitations that came with using real devices, such as not being able to access the router of our network, we opted for simulation.
In this chapter, we will discuss how we used Mininet~\citep{lantz_mininet_2021}, a realistic virtual network, in our evaluation.

Mininet is a tool that network developers and researchers can use to create software-defined networks (SNDs) using the $OpenFlow$ standard.

\begin{figure}[ht]
    \centering
    \includegraphics[width=0.5\linewidth]{images/mininet_topo.png}
    \caption{The resulting Dumbbell topology with 5 hosts on either side of the switch. This topology simulates congestion on the link as the hosts have to share it for their data transfer.}
    \label{fig:mininet-topo}
\end{figure}

Using the Python API provided, we created the network topology shown in Figure~\ref{fig:mininet-topo}.
The script takes several parameters to create a simulation environment resembling a realistic scenario.
The variables that the script changes between simulations are the link's \textit{bandwidth}, \textit{delay} and the rate of \textit{packet loss}.

The bandwidth of a link is the maximum rate of data transfer we can achieve.
In contrast to bandwidth in signal processing, we measure bandwidth in bits per second rather than hertz in computer networking.
The delay of a link specifies the latency of the link.
It is the time that a bit of data takes to travel across a link. 
We measure this in milliseconds.
Link delay corresponds to the geographical distance between the communicating parties; however, in the case of IoT, we can expect devices to be in local proximity.
Lastly, the packet loss rate shows the percentage of corrupted or dropped packets in transit.
Various protocols having to retransmit packets also adds to the delay of data transfer.
Importantly, we have only considered the typical circumstances of packet loss and have not included scenarios such as interference or packet loss attacks.

The bandwidth and delay numbers correspond, as closely as possible, to various link types in a network.
To do so, we have gathered data from the~\cite{ofcom_uk_2021} report on UK broadband speeds.
There were specific cases in which it was not possible to find this data in the report; hence it was augmented using a similar methodology in work conducted by~\cite{previdi_is-is_2019} and in the case of ZigBee, the work by~\citet{alena_fault_2011}.

\begin{table}[ht]
    \caption{The parameters chosen for each link simulation in Mininet. The types of links were chosen as the most commonly occurring ones in IoT use cases. The data also assumes a typical IoT setup where most devices are within local geographical proximity. That is, the devices are communicating with each other within the range of one factory or site, with only the central node communicating with some server.}\label{tab:links}
    %\tt 
    \rowcolors{2}{}{gray!3}
    \begin{tabular}{@{}llll@{}}
        \toprule
        \textbf{Simulated Link Type} & \textbf{Link bandwidth (Mb/s)} & \textbf{Link delay (ms)} & \textbf{Packet loss rate (\%)} \\
        Wi-Fi                        & \texttt{30}                    & \texttt{10}              & \texttt{2}                     \\
        ZigBee                       & \texttt{0.25}                  & \texttt{5}               & \texttt{1}                     \\
        4G                           & \texttt{4}                     & \texttt{20}              & \texttt{1.5}                   \\
        3G                           & \texttt{1}                     & \texttt{40}              & \texttt{1.5}                   \\
        100Mb Ethernet               & \texttt{100}                   & \texttt{1}               & \texttt{0.2}                   \\
        \bottomrule
    \end{tabular}
\end{table}

It was complicated to find exact estimates for packet loss rates, with most sources describing approximations for a stable connection~\citep{sdu_ictp-sdu_2013} and not precise measurements.
Hence, the data are best estimates, cross-validated through the different sources and are not exact values.

Using the different links, we then transferred a file of equal size using the various QUIC implementations described in Chapter~\ref{chap:quic_impl} for the evaluation of QUIC implementations.

We evaluated the MQTT/QUIC implementation performance using the same topology and simulation parameters.
In general, MQTT allows for messages with a maximum size of approximately 260MB.
However, this is a huge message, and most publicly deployed brokers will reject it, so a general use-case was simulated.

Each topic in MQTT consists of a hierarchy of topic levels separated by a forward slash.
For example, in a smart home scenario, we may have a topic like $home/groundfloor/kitchen/temp$ to control the temperature in the kitchen via a smart thermostat.
A topic may also include a wildcard.
The topic string $home/groundfloor/+/temp$ includes a \textit{single-level} wildcard that will match an arbitrary string.
This would match the topic $home/groundfloor/lounge/temp$, but not match the topic $home/secondfloor/kitchen/temp$.
If a client wishes to subscribe to multiple topics with the same prefix, a \textit{multi-level} wildcard may be used.
For example, the topic $home/secondfloor/kitchen/\#$ can be used to subscribe to all topics with a prefix matching the string before the hash character.
Notably, brokers reserve topics for system messages starting with the \$ character.

Taking this into account, we opted for a smart home scenario to simulate the MQTT communication.
The data transmitted can be found in Appendix~\ref{appendix:mqtt_message}.
\section{Connection time comparison} \label{sec:conn_time}
\section{Data transmission comparison}

We next evaluate the time it took both implementations to transmit the aforementioned MQTT messages.
To do so, we have used the previously described methodology.

Based on the analysis of connection time and the previous discussion, we hypothesise that:

\begin{itemize}
    \item \textbf{H1}: $MQuicTT$ performs on-par with $rumqtt$ in the IoT home scenario but does not present a significant difference.
    \item \textbf{H2}: $MQuicTT$ performs significantly better than $rumqtt$ in the printer farm scenario.
    \item \textbf{H3}: $MQuicTT$ can transmit all the messages on all data links in the synthetic scenario, whereas $rumqtt$ is not.
\end{itemize}

Figure~\ref{fig:comm_time} shows the results of this experiment.

\begin{figure}[ht]
    \centering
    \includegraphics[width=1\linewidth]{images/analysis_comm_time.png}
    \caption{The time it took for the implementations to transmit the MQTT messages.
        The base $rumqtt$ implementation is shown in green and $MQuicTT$ in orange.
        Each column represents a scenario, and each row is a data link.
        The time for each Figure is shown in milliseconds. The base implementation of $rumqtt$ could not transmit data in the synthetic packet loss scenario.}
    \label{fig:comm_time}
\end{figure}

We can see that the results support hypothesis \textbf{H1}.
That is,  $MQuicTT$ performs almost identically to $rumqtt$ across all data links for the IoT home scenario.
In fact, $MQuicTT$ actually performs marginally better than $rumqtt$ in this scenario with the transmission time of $MQuicTT$ being on average $2.02\%$ lower than that of $rumqtt$.
From this we can see that $MQuicTT$ presents no disadvantages in terms of transmission time even in network with low packet loss and congestion.

Hypothesis \textbf{H2} however is not supported.
The performance advantage in terms of transmission time for $MQuicTT$ in the printer farm scenario is almost identical to that of the smart home scenario.
On average, $MQuicTT$ transmits the data $2.36\%$ faster than $rumqtt$ and although this is an improvement, it is comparable to the improvement in the IoT home scenario.

The result means that the presented use-case does not present a high enough packet loss percentage for $MQuicTT$ to have a significant advantage.
This can be verified by looking at the synthetic scenario, which validates hypothesis \textbf{H3}.
In the synthetic scenario only $MQuicTT$ managed to transmit the data on all data links and transmitted the data $13.8\%$ faster using ethernet.
Hence, in environments that may experience extreme data loss, $MQuicTT$ presents a clear advantage, however, finding such a scenario may be difficult.

Overall, $MQuicTT$ transmitted the data faster than $rumqtt$ across all scenarios and has shown to be more resilient against high packet loss.
It may however be ineffective overhead to migrate existing deployments to it due to the marginal benefits in low congestion environments.

\section{Binary size experiment design} \label{sec:exp_bin}

The last section of this chapter describes how we have conducted further analysis on the binary size of the QUIC protocol that underlines MQuicTT.
The focus of this section is to see if the QUIC stack contributes a significant overhead to MQuicTT's binary size and how we can reduce this overhead.
We expect that the QUIC stack will contribute to the majority of the size of MQuicTT as MQTT itself is designed to have a low code size overhead.
Hence, the first step of this part of the analysis will be to determine how much QUIC contributes to the binary size.

In order to get a breakdown of the binary we have used the $cargo-bloat$~\footnote{cargo-bloat - \url{https://lib.rs/crates/cargo-bloat}} utility.
The utility analyses the binary using custom ELF, DWARF and Mach-O parsers and disassembles the binary to look for references and links to anonymous data.
Doing so creates a map of the binary that shows where every byte has a label attached to it.

This utility provides the composition of a Rust binary. 
However, it is not perfect and results in some margin of error.
Unfortunately, this margin of error is also not easily measurable.
By comparing the total size of the binary as reported by $cargo-bloat$ to the size reported by the operating system, we have deduced that the total error margin is within $1\%$ with good precision.
This should mean that we can get a somewhat accurate error margin on the components. 
However, it is also possible that the internal calculations are inaccurate despite the overall size being accurate.

The next step in this stage will be analysing methods for trimming down the QUIC stack binary size.
Hence, in this stage, we shift our focus to the binary produced by $QuicSocket$.
To reduce the binary size, we opt to use the method established by~\citet{eggert_towards_2020} as recreating these steps may show a general framework for reducing binary sizes for hardware constrained devices.
Notably, our application already handles client and server code separately; the MQTT broker requires a different binary to the MQTT client.

Hence, the steps we take are as follows:

\begin{itemize}
    \item Compile the binary for a 32-bit target by setting the $target$ flag in cargo to $i686-unknown-linux-gnu$.
    \item Remove any error handling code beyond what is needed for the binary to compile.
    \item Remove any code that writes to standard output.
\end{itemize}

After every step, we record the difference in binary size made by the change using the same methodology.

Once this step is completed we further analyse the size of $Quinn$ and $Rustls$ using a by-function binary size breakdown.
Using the $cargo-bloat$ utility we can get a list of the contribution of each function to the binary size and then assign each function to its respective protocol feature.

\section{Binary size breakdown} \label{sec:binary_sizes}

We first look at an in-depth breakdown of the composition of the binary of the underlying QUIC implementation.
As previously established, hardware constrained devices do not have much space for firmware; hence, identifying sections of the QUIC stack that can be trimmed down or eliminated entirely is essential.
In order to get a breakdown of the binary we have used the $cargo-bloat$~\footnote{cargo-bloat - \url{https://lib.rs/crates/cargo-bloat}} utility.

[figure of results here]

[discussion of results here]

In order to reduce the binary size we opt to use the method established by~\citet{eggert_towards_2020} as recreating these steps may show a general framework for reducing binary sizes for hardware constrained devices.
Notably, our application already handles client and server code separately; the MQTT broker requires a different binary to the MQTT client.

Hence, the steps we have taken are as follows:

\begin{itemize}
    \item Compile the binary for a 32-bit target by setting the $target$ flag in cargo to $i686-unknown-linux-gnu$.
    \item Remove any error handling code beyond what is needed for the binary to compile.
    \item Remove any code that writes to standard output.
\end{itemize}

The resulting binary sizes after these steps can be found in Figure TBA.

[Figure here]

As we can see, after trimming down the binary, the TLS implementation stands out as the most considerable dependency.
While~\cite{eggert_towards_2020} attempts to address this by creating minimal cypher implementations used by TLS, we instead opt to discuss the possibility of a complete alternative to TLS, suitable for hardware constrained devices in Section~\ref{chap:TLS}.


\subsection{Summary}

When discussing the possibility of using a Rust QUIC implementation as the transport layer protocol for network firmware in IoT devices we established that the QUIC implementation must perform at least as well as the baseline and have a binary size which can fit onto widely used IoT devices.

From the above analysis we can conclude that $MQuicTT$ performs at least as well as the baseline TCP implementation of $rumqtt$.
When it comes to connection time, $MQuicTT$ performs marginally better, and when it comes to total transmission time, the implementations are on par.
Hence, this requirement is satisfied.

In terms of binary size we have managed to reduce it to approximately $8Mb$.
This size of binary would easily be installable on popular IoT devices such as the Raspberry Pi 3 Model B, Beagle Board or the Arduino Due.
However, this size of binary would not support industrially-wide used chips such as the esp32.
We have analysed the possible avenues of further reducing the binary size of $MQuicTT$ by addressing issues with the regex library and by analysing $Quinn$ and $Rustls$ by feature.
Hence, overall, we can say that we have achieved creating an implementation that can be used on a large number of IoT devices, but not on ones with stricter hardware constraints.
