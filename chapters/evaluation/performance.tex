\section{Data transmission comparison}

We next evaluate the time it took both implementations to transmit the aforementioned MQTT messages.
To do so, we have used the previously described methodology.

Based on the analysis of connection time and the previous discussion, we hypothesise that:

\begin{itemize}
    \item \textbf{H1}: $MQuicTT$ performs on-par with $rumqtt$ in the IoT home scenario but does not present a significant difference.
    \item \textbf{H2}: $MQuicTT$ performs significantly better than $rumqtt$ in the printer farm scenario.
    \item \textbf{H3}: $MQuicTT$ can transmit all the messages on all data links in the synthetic scenario, whereas $rumqtt$ is not.
\end{itemize}

Figure~\ref{fig:comm_time} shows the results of this experiment.

\begin{figure}[ht]
    \centering
    \includegraphics[width=1\linewidth]{images/analysis_comm_time.png}
    \caption{The time it took for the implementations to transmit the MQTT messages.
        The base $rumqtt$ implementation is shown in green and $MQuicTT$ in orange.
        Each column represents a scenario, and each row is a data link.
        The time for each Figure is shown in milliseconds. The base implementation of $rumqtt$ could not transmit data in the synthetic packet loss scenario.}
    \label{fig:comm_time}
\end{figure}

[Discussion here]