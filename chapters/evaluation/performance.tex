\section{Data transmission comparison}

We next evaluate the time it took both implementations to transmit the aforementioned MQTT messages.
To do so, we have used the previously described methodology.
That is, we have transmitted the MQTT messages shown in~\ref{appendix:mqtt_message} for their respective scenario and have measured the total time taken for the transmission to occur.
We have repeated this and taken the average of the results to ensure that they are representative.

Based on the analysis of connection time and previous discussions, we hypothesise that:

\begin{itemize}
    \item \textbf{H1}: $MQuicTT$ performs on-par with $rumqtt$ in the IoT home scenario but does not present a significant difference.
    \item \textbf{H2}: $MQuicTT$ performs significantly better than $rumqtt$ in the printer farm scenario.
    \item \textbf{H3}: $MQuicTT$ can transmit all the messages on all data links in the synthetic scenario, whereas $rumqtt$ is not.
\end{itemize}

Figure~\ref{fig:comm_time} shows the results of this experiment.

We can see that the results support hypothesis \textbf{H1}.
That is,  $MQuicTT$ performs almost identically to $rumqtt$ across all data links for the IoT home scenario.
In fact, $MQuicTT$ actually performs marginally better than $rumqtt$ in this scenario with the transmission time of $MQuicTT$ being on average $2.02\%$ lower than that of $rumqtt$.
From this we can see that $MQuicTT$ presents no disadvantages in terms of transmission time even in network with low packet loss and congestion.
For new deployments it would be advantages to use $MQuicTT$, however, for existing ones, it may not be advisable to migrate deployments.

Hypothesis \textbf{H2} however is not supported.
The performance advantage in terms of transmission time for $MQuicTT$ in the printer farm scenario is almost identical to that of the smart home scenario.
On average, $MQuicTT$ transmits the data $2.36\%$ faster than $rumqtt$ and although this is an improvement, it is comparable to the improvement in the IoT home scenario.
Hence, the results draw a similar conclusion to that of the IoT home scenario.
It is still advantages to use $MQuicTT$ for new deployments.

The results of the printer farm scenario show that the presented use-case does not present a high enough packet loss percentage for $MQuicTT$ to have a significant advantage.
This can be verified by looking at the synthetic scenario, which validates hypothesis \textbf{H3}.
In the synthetic scenario only $MQuicTT$ managed to transmit the data on all data links and transmitted the data $13.8\%$ faster using ethernet.
Hence, in environments that may experience extreme data loss, $MQuicTT$ presents a clear advantage, however, finding such a scenario may be difficult.

Overall, $MQuicTT$ transmitted the data faster than $rumqtt$ across all scenarios and has shown to be more resilient against high packet loss.
It may however be ineffective overhead to migrate existing deployments to it due to the marginal benefits in low congestion environments.